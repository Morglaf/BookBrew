\documentclass{article}
\usepackage{pdfpages}
\usepackage[margin=0cm, paper=a3paper, landscape, top=5cm]{geometry}
\usepackage{fontspec}
\usepackage[french]{babel}
\usepackage{microtype}
\usepackage{graphicx}
\usepackage{titlesec}
\usepackage{xcolor}
\usepackage{float}
\usepackage{tikz}

% Utiliser Garamond comme police principale
\setmainfont{Garamond}[Scale=1.10]
\newfontfamily\titrefont{Garamond}
\newfontfamily\headerfont[Scale=0.62]{Garamond}

% Commande pour redimensionner les images à la largeur maximale de la page
\setkeys{Gin}{width=\linewidth, keepaspectratio}

% Environnement personnalisé pour centrer les images
\floatstyle{plain}
\newfloat{inlinefigure}{H}{lof}
\floatname{inlinefigure}{Figure}

\let\oldincludegraphics\includegraphics
\renewcommand{\includegraphics}[2][]{%
  \begin{inlinefigure}
    \centering
    \oldincludegraphics[#1]{#2}
  \end{inlinefigure}%
}

\newcommand{\tranche}{{{spineThickness}}}

\definecolor{lightgray}{gray}{0.9} % Define light gray color

\begin{document}
% First side
\begin{figure}[ht!]
  \centering
  \begin{minipage}[t][210mm][t]{143.5mm}
    \vspace{0.0mm} % Adjust this value to correct the height difference
    \colorbox{lightgray}{% Add this line
      \begin{minipage}[t][210mm][t]{\linewidth}
        \vspace{10mm} % Top margin
        \hspace{10mm} % Left margin
        \centering
        {\Large\titrefont {{titre}}}\\
        {{auteur}}\\
      \end{minipage}%
    }% Add this line
  \end{minipage}%
  \hspace{\tranche}
  \begin{minipage}[t][210mm][t]{143.5mm}
    \vspace{0.0mm} % Adjust this value to correct the height difference
    \colorbox{lightgray}{% Add this line
      \begin{minipage}[t][210mm][t]{\linewidth}
        \vspace{10mm} % Top margin
        \hspace{10mm} % Left margin
        \centering
        {{edition}}\\
      \end{minipage}%
    }% Add this line
  \end{minipage}%
\end{figure}

% Ajouter les traits de coupe
\begin{tikzpicture}[remember picture,overlay]
    % Taille des traits de coupe
    \newcommand{\cutsize}{10mm}

    % Traits de coupe aux coins
    \draw[thick] (current page.south west) ++(0,\cutsize) -- ++(\cutsize,0);
    \draw[thick] (current page.south west) ++(\cutsize,0) -- ++(0,\cutsize);

    \draw[thick] (current page.south east) ++(0,\cutsize) -- ++(-\cutsize,0);
    \draw[thick] (current page.south east) ++(-\cutsize,0) -- ++(0,\cutsize);

    \draw[thick] (current page.north west) ++(0,-\cutsize) -- ++(\cutsize,0);
    \draw[thick] (current page.north west) ++(\cutsize,0) -- ++(0,-\cutsize);

    \draw[thick] (current page.north east) ++(0,-\cutsize) -- ++(-\cutsize,0);
    \draw[thick] (current page.north east) ++(-\cutsize,0) -- ++(0,-\cutsize);

    % Traits de coupe au centre des bords, en tenant compte de \tranche
    \draw[thick] (current page.south) ++({0-(\tranche)/2},\cutsize) -- ++(0,-2*\cutsize);
    \draw[thick] (current page.south) ++({0+(\tranche)/2},\cutsize) -- ++(0,-2*\cutsize);

    \draw[thick] (current page.north) ++({0-(\tranche)/2},-\cutsize) -- ++(0,2*\cutsize);
    \draw[thick] (current page.north) ++({0+(\tranche)/2},-\cutsize) -- ++(0,2*\cutsize);


\end{tikzpicture}

\end{document}
